\documentclass[]{article}

\usepackage[letterpaper]{geometry}

\usepackage{xr}
\externaldocument{paper}

\RequirePackage{amsmath,amsfonts,amssymb}
%\RequirePackage[authoryear]{natbib}
\usepackage{longtable}
%\usepackage[notablist,tablesfirst]{endfloat}% To `separate' figures and tables from text if required

\usepackage{epstopdf}% To incorporate .eps illustrations using PDFLaTeX, etc.
\usepackage[caption=false]{subfig}% Support for small, `sub' figures and tables
\usepackage[doublespacing]{setspace}% To produce a `double spaced' document if required
%\setlength\parindent{24pt}% To increase paragraph indentation when line spacing is doubled

\usepackage[longnamesfirst,sort]{natbib}% Citation support using natbib.sty
\bibpunct[, ]{(}{)}{;}{a}{,}{,}% Citation support using natbib.sty
\renewcommand\bibfont{\fontsize{10}{12}\selectfont}% To set the list of references in 10 point font using natbib.sty


% \theoremstyle{plain}% Theorem-like structures provided by asthma.sty
% \newtheorem{theorem}{Theorem}[section]
% \newtheorem{lemma}[theorem]{Lemma}
% \newtheorem{corollary}[theorem]{Corollary}
% \newtheorem{prop}[theorem]{Proposition}

% \theoremstyle{definition}
% \newtheorem{definition}[theorem]{Definition}
% \newtheorem{example}[theorem]{Example}

% \theoremstyle{remark}
% \newtheorem{remark}{Remark}
% \newtheorem{notation}{Notation}

\usepackage{amsthm}
\usepackage[utf8]{inputenc} % set input encoding (not needed with XeLaTeX)
\usepackage{bm}
\usepackage[hidelinks]{hyperref}
\usepackage{graphicx}
\usepackage{mathtools}
%\usepackage{fullpage}
\usepackage{makecell}
\usepackage{booktabs}
\usepackage{multirow}
\usepackage{numdef}
%\usepackage{multibib}
\usepackage{rotating}
\usepackage{xspace}
\usepackage[dvipsnames]{xcolor}

%\startlocaldefs

\newcommand{\rd}[1]{\textcolor{BrickRed}{#1}}

\newcommand{\reg}[2]{g\left(#1;#2\right)}
\newcommand{\regc}[2]{g_C\left(#1;#2\right)}
\newcommand{\regt}[2]{g_T\left(#1;#2\right)}
\newcommand{\bx}{\bm{x}}
\newcommand{\bxi}{\bm{x}_i}
\newcommand{\bxs}{\bx}%{\bx^S}
\newcommand{\bxy}{\bx^Y}
\newcommand{\bxyt}{\bx^{Y\prime}}
\newcommand{\bxsi}{\bxs_i}
\newcommand{\bxsit}{\bm{\tilde{x}^S}_i}
\newcommand{\bxsitp}{\bm{\tilde{x}^{S\prime}}_i}
\newcommand{\EE}{\mathbb{E}}
\newcommand{\pp}{e(\bxs)}
\newcommand{\ppi}{e(\bxsi)}
\newcommand{\ppip}[1]{e_{#1}(\bxsi)}
\newcommand{\hpp}{\hat{e}(\bxs)}
\newcommand{\hppi}{\hat{e}(\bxsi)}
\newcommand{\ppf}[2]{e_{#1}\left(#2\right)}
\newcommand{\yt}{Y_T}
\newcommand{\yc}{Y_C}
\newcommand{\yti}{Y_{Ti}}
\newcommand{\yci}{Y_{Ci}}
\newcommand{\sti}{S_{Ti}}
\newcommand{\sci}{S_{Ci}}
\newcommand{\st}{S_T}
\num\newcommand{\eff0}{\tau^0}
\num\newcommand{\eff1}{\tau^1}
\num\newcommand{\mut1}{\mu_T^1}
\num\newcommand{\mut0}{\mu_T^0}
\num\newcommand{\muc1}{\mu_C^1}
\num\newcommand{\muc0}{\mu_C^0}
\num\newcommand{\heff0}{\hat{\tau}^0}
\num\newcommand{\heff1}{\hat{\tau}^1}
\newcommand{\mucs}{\mu_C^s}
\newcommand{\muts}{\mu_T^s}
\newcommand{\effs}{\tau^s}
\newcommand{\ri}{R_i}
\newcommand{\blam}{\bm{\Lambda}}
\newcommand{\dblam}{\bm{\dot{\Lambda}}}
\newcommand{\blamh}{\bm{\hat{\Lambda}}}
%\newcommand{\r}{R}


\newcommand\independent{\protect\mathpalette{\protect\independenT}{\perp}}
\def\independenT#1#2{\mathrel{\rlap{$#1#2$}\mkern2mu{#1#2}}}


\theoremstyle{plain}
\newtheorem{prop}{Proposition}
\newtheorem{lemma}{Lemma}

\theoremstyle{remark}
 \newtheorem{ass}{Assumption}


%\endlocaldefs
%\newcites{supp}{Supplementary References}

%\begin{frontmatter}


\newenvironment{itquote}
  {\begin{quote} \itshape}
  {\end{quote}\ignorespacesafterend}


\title{Responses to Review Comments}
%\runtitle{\textsc{geepers}}
%\maketitle
\date{}

\begin{document}

\maketitle
We are grateful for the reviewers' and the editor's careful reading
and comments. We believe that the the substantial changes we made
chages to the manuscript in response to those comments greatly
improved the paper. Most significantly, we clarified our presentation
and added a second application.
We provide point-by-point responses below.

\section*{Reviewer 1}

The article effectively presents a simulation study comparing three methods and provides a clear application example of applying GEEPERs. However, there are notable areas for improvement, including a lack of literature review on previous simulation studies about Principal Stratification and some unclear presentations of results. Specific comments are outlined below.

\begin{enumerate}
 
\item    \begin{itquote}  Page 3, Line 10: The rationale for using Principal Stratification to estimate treatment effects for bottom-outers is insufficient. Additionally, it is advisable to compare the Principal Stratification method to other causal inference methods, such as instrumental variable and propensity score analysis, to provide a comprehensive understanding of the chosen approach.
\end{itquote}

We added Section \ref{sec:psInEdTech} to clarify the role principal stratification can play in analyzing implementation data---such as records of which students requested bottom-out hints---from an RCT studying educational effectiveness. 
In that section, we also compared principal stratification to other relevant methods, as you suggested. 

\item \begin{itquote}
    
     Page 5, Line 5: While the authors effectively describe the novelty of their study, the literature review lacks the historical development of Principal Stratification. Including more in-depth reviews and relevant citations would enhance the background information and context of the study.
\end{itquote}

Thanks for this suggestion. We have added material to the introduction, starting from the paragraph beginning ``The conceptual basis'' on page \pageref{background} and through the following page, describing some of the history and background of principal stratification. 

\item \begin{itquote}
     Page 15: The simulation details about how variable Z is simulated and the number of replications for each condition are not clearly explained. Providing more information on these aspects would contribute to the transparency and replicability of the study.
\end{itquote}
We agree that the previous draft was unclear in the description of the outcome-generating model (now equation \ref{eq:y-sim}). We have added material to section \ref{sec:dataGeneration} and Table \ref{tab:factor} that includes the necessary details. Should anything still be unclear, replication code is available in the anonymized GitHub repository at \url{https://anonymous.4open.science/r/psGee-B498/}.


\item \begin{itquote}     Page 17: The rationale for selecting specific values for manipulated factors, such as choosing a sample size range of 100 to 1000 and using N(0,1/2) instead of N(0,1), is not adequately justified. Including evidence and citations to support these choices would strengthen the methodological foundation of the study.

\end{itquote}
We added rationale, including citations, to the paragraph beginning ``Potential and observed outcomes...'' on page \pageref{potential} and to section \ref{sec:manipulatedFactors}.

\item \begin{itquote}     Page 18: A section detailing the evaluation metrics and criteria used to assess simulation results should precede the presentation of results. This would provide readers with a clear understanding of the benchmarks against which the findings are measured.

\end{itquote}
We now begin Section \ref{sec:simResults} with definitions of the evaluation metrics, which we hope make the results clearer.

\item \begin{itquote}     Pages 18-20: It is highly recommended that the authors employ a series of ANOVAs to analyze the impact of various factors on the accuracy of effect estimation to guide results interpretation. Additionally, a concise paragraph summarizing key findings before the application example would enhance the clarity and interpretation of the results.
\end{itquote}
Based on your comment, we strongly considered including \textsc{anova} results to gauge the factors' relative importance. However, we decided that such an analysis would not be appropriate in this context. First, the factors were not fully crossed, significantly complicating the analysis. Second, the results would be heavily driven by the factors' variances---specifically, the set of values we chose for $n$, $\alpha$, $\bm\beta$, and $\bm\gamma$; although our choices were well-motivated, they were also, necessarily, somewhat arbitrary. 
Lastly, as Figure \ref{fig:alphan} shows, the effects of $n$ and $\alpha$ on bias and standard error are non-linear, further complicating the \textsc{anova} modeling. For those reasons, we decided our current approach was clearer and simpler.

Your suggestion of a paragraph summarizing the results was quite helpful. We included such a paragraph, on page \pageref{simsum} beginning ``In sum.''

\item \begin{itquote}      APA Citations and References: The paper does not adhere to the APA 7th style. Authors are advised to consult the APA Formatting and Style Guide for proper citation and reference formatting throughout the manuscript. Consistency in style is crucial for scholarly publications.
\end{itquote}
Thank you. We have fixed the formatting.
\End{enumerate}

\section*{Reviewer 2}
Comments to the Author
Great work. Well written. Please find some of my comments below. I hope they are useful.

\begin{enumerate}

\item \begin{itquote}   Please provide more background on M-estimators especially to the extent it evolves to become relevant to the GEEPERs. 
\end{itquote}
Thanks for this suggestion---in response, we added Section \ref{sec:mest}
\item \begin{itquote}      I am wondering if doubly robust PSW estimator would reduce the remaining bias difference between GEEPERs and PSW, and Mixture and PSW.
\end{itquote}
This is an interesting suggestion---you're right that the \psw estimator did not include an outcome model, unlike \geepers and \pmm. However, we doubt that an outcome model would substantially reduce the bias of \psw in the simulation. The advantage of doubly-robust estimation is that an outcome model can correct for model misspecification in the propensity (or principal) score model. However, the principal score model in our simulation study was well-specified---the bias of the \psw estimator was due to an unobserved confounder, $x_3$. Unfortunately, like \psw, doubly-robust estimators rely on ignorability, i.e., that all relevant confounders have been measured and included. 
\item \begin{itquote}      GEEPERs seems great, but also, surprised by how efficient Mixture can be. Are there any mechanisms that link M-estimators to Mixture modeling? Just wondering.
\end{itquote}
We were also surprised by how well the \pmm performed in the simulation, especially given prior literature detailing its vulnerability \citep[][; cited in the paper]{griffin2008application,feller2016principal}. Its similarity to \geepers in some scenarios is less surprising, since they are both essentially mixture model estimators.

\item \begin{itquote}    Would it be possible to offer some evaluation metric for GEEPERs, which would indicate, under some conditions researchers could possibly opt or seek other estimators?
\end{itquote}
This is a great suggestion. We added the paragraph beginning ``The simulation study suggests\dots'' on page \pageref{evalMetric} to give some guidance along those lines. 
\item \begin{itquote}     Quickly checked Frangakis and Rubin (2002). Do you think the effect of bottom-outers (or similar experimental designs) could be partitioned into associative and dissociative effects with GEEPERs?
\end{itquote}
This is correct, but only in a narrow technical sense. 
Under our notation, Frangakis and Rubin (2002) define dissociative effects as contrasts between $\yti$ and $Y_{Ci}$ for the subjects $i$ with $S_{Ci}=\sti$; associative effects are contrasts for subjects with $S_{Ci}\ne\sti$. A contrast between associative and dissociative effects ``measures the degree to which a causal effect of treatment on outcome occurs together with a causal effect of treatment on the surrogate."  Since our paper deals only with the one-way noncompliance case, we set $S_{Ci}=0$ for all $i$ (Assumption 1). In Section \ref{sec:fh2t}, we define $\st=0$ for non-bottom-outers and $\st=1$ for bottom-outers. Therefore, as you wrote, an effect for non-bottom-outers would be dissociative, while an effect for bottom-outers would be associative. However, this is, at least in part, an artifact of our notation. It's true that anyone assigned to control would, by necessity, request fewer than 11 bottom-out hints (the median in the treatment group)---$S_C=0$---and that treatment assignment affects bottom-out-hint usage less for non-bottom-outers than for bottom-outers. However, many non-bottom-outers assigned to treatment will request \emph{some} bottom-out hints, whereas students assigned to control request none---our choice to dichotomize hint requests masks variation that does not exist in the control group. Furthermore, our analysis would have been identical had we set $S_{Ci}$ equal to any other constant or set it to NA. 

\item \begin{itquote}    It would be great if you could share the code for estimation and simulation in the next round of review (if there is). Also make sure you have comments in the code so someone reading the article can match the narrative with the code.
\end{itquote}
Full replication code is available at \url{FILL IN} 
\end{enumerate}

\section{Reviewer 3}


\subsection{Simulations}
\begin{itquote}
    In my view, the simulations should show the following:
\begin{itemize}
   \item When the assumptions of extant methods are met, GEEPERS should do no worse.
   \item When the assumptions of extent methods are not met, GEEPERS should do better.
\end{itemize}
\end{itquote}
Depending on the meaning of ``do no worse,'' \geepers may or may not meet those criteria. 
In the scenarios we tested, \geepers is always less biased and its nominal confidence intervals achieve higher empirical coverage than the competitors we tested.
However, the competing methods often boast smaller standard errors than \geepers. 
This set of results is typical for semi- or non-parametric methods, which tend to be less efficient than their parametric counterparts when the latter's assumptions are met, but are, say, less biased when those assumptions are violated.

Unfortunately, \geepers may not even meet that criterion relative to \psw. Our simulation did not include scenarios in which principal ignorability held, but we expect that in such scenarios---depending on other factors---\psw may outperform \geepers in every respect. 
That said, principal ignorability is untestable and, in our opinion, rather strong. 
As we stated on page \pageref{evalMetric} (in the paragraph beginning ``The simulation study suggests''), researchers will have to decide in each application which assumptions are the most plausible, and choose estimators based on that decision. 

\begin{itquote} Given this criterion, it seems to me that the simulation should focus on the last three factors in the design.  That is, I would hold the sample size and alpha fixed.  Currently, too many factors are varied at the same time to give a clear picture.  
I would alter Table 1 to indicate which factors yield assumption violations and then violate these assumptions in isolation instead of relaxing them simultaneously.  As it currently stands, it is quite hard to understand the relative performance of GEEPERs compared to extant methods. Careful relaxation of each assumption allows the reader to see how much bias results from a single factor. 
\end{itquote}
The simulation has two parts: in the first part, we estimate the bias and standard error of \geepers and \pmm for a range of values of $\alpha$ and $n$ when the assumptions of both methods are met.
The goal for the first part is to determine the required sample size and covariate predictability necessary for \geepers to perform well. 

The second part presents results for scenarios where assumptions are violated. 
The results in Figure \ref{fig:boxplots} and in Table \ref{tab:coverage} now hold $n$ fixed at $n=500$ per group and completely cross the other factors, i.e. $\alpha=0.2$ or $0.5$, interactions between $\bm{x}$ and $\st$ or $Z$, and the distribution of the residuals.
Now it is apparent from Figure \ref{fig:boxplots} that, for instance, none of the methods are sensitive to interactions between $\bm{x}$ and $\st$, and that \pmm, but not the others, is sensitive to the distribution of the residuals.  

Following your fine suggestion, we formatted Table \ref{tab:coverage}, highlighting results in red when one or more assumptions have been violated.
This format makes it clearer that, for instance, \pmm makes distributional assumptions while \geepers does not, and that \geepers is less sensitive to the assumptions that both estimators make, especially when $\alpha=0.5$. 


\subsection{Application}
\begin{itquote}
I thought there were a number of less-than-desirable parts to the application. First, the focus on bottoming out is less than ideal. This strikes me as a fairly rare phenomenon. I didn't find the concept very intuitive and had to read just the descriptive section several times to understand why one might be interested in a principal stratification analysis here.  This makes the methods seem not particularly general.  Why not focus on an application with straightforward one-way noncompliance? This is an area where principal stratification is often applied, and the concepts are well understood.

In addition, it isn't ideal that the ATE in each condition is not estimated very precisely; that is, the CI appears to cover zero in all cases. Next, all the principal stratification estimates are pretty close to the same value as the ATE. That serves as another reason this application isn't compelling. It is hard to care about an application where principal stratification doesn't provide much in the way of insights relative to the ATE. Next, the performance of GEEPERs seems to be quite similar to extant methods in each scenario. Overall, to my thinking, there is little to recommend the current application. Again, there is little here that demonstrates the value of GEEPERs.
\end{itquote}

These are important points, which we attempted to address in two ways. 
Most importantly, we included an additional application example, a dental clinical trial with noncompliance, in which \geepers estimated statistically significant principal effects. This example is clean and straightforward, showing an instance where \geepers yields interesting results. 
However, our substantive interests and expertise, as well as those of most of the JREE readership, lie in education research rather than dental science, which prevents us from offering a comprehensive discussion of the method's application or interpreting the \emph{a priori} surprising results. 

For that reason, we retained the original application. 
However, we tried to clarify the role PS could play in analyzing student implementation in an RCT studying educational technology. 
For instance, in the Introduction, we wrote ``Indeed, if the use of bottom-out hints is considered part of the treatment, then those students who rarely used them are partial compliers---could the treatment still benefit them?''
Most prominently, we added an extensive discussion in Section \ref{sec:psInEdTech}.


    
\end{document}}